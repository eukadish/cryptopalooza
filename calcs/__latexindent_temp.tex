\documentclass{article}

% set font encoding for PDFLaTeX, XeLaTeX, or LuaTeX
% \usepackage{ifxetex,ifluatex}
% \if\ifxetex T\else\ifluatex T\else F\fi\fi T%
%   \usepackage{fontspec}
% \else
%   \usepackage[T1]{fontenc}
%   \usepackage[utf8]{inputenc}
%   \usepackage{lmodern}
% \fi

% \usepackage{hyperref}

% \title{Title of Document}
% \author{Name of Author}

% Enable SageTeX to run SageMath code right inside this LaTeX file.
% http://doc.sagemath.org/html/en/tutorial/sagetex.html
% \usepackage{sagetex}

% Enable PythonTeX to run Python – https://ctan.org/pkg/pythontex
% \usepackage{pythontex}


\pagestyle{empty}

\usepackage[margin = 2.0 cm]{geometry}

\usepackage{amsmath}

\begin{document}
% \maketitle

\subsection*{QAP (Quadratic Arithmetic Program)}

\noindent Sample calculations in deriving some of the polynomials used in
constructing a \textit{SNARK} for the equation $ 3 * x = 6 $. The equation

\[
    (v_{0} + \Sigma_{k = 1}^{2} a_{k} v_{k}) (w_{0} + \Sigma_{k = 1}^{2} a_{k} w_{k}) - (y_{0} + \Sigma_{k = 1}^{2} a_{k} y_{k})
\]

\noindent is guaranteed to have the form $ h(x) t(x) = h(x) (x - r) $ by the
\textit{Polynomial remainder theorem} and more specifically the
\textit{Factor theorem}. The first set of polynomials are the equations:

\begin{align*}
    v^{'}_{0}(x) & = 1 * \frac{(x  -  r) (x  - r_{1}) (x - r_{2}) (x  - s_{2})}{(s_{1} -  r) (s_{1} - r_{1}) (s_{1} - r_{2}) (s_{1} - s_{2})} \\
    & \qquad \qquad + 1 * \frac{(x  -  r) (x  - r_{1}) (x  - r_{2}) (x  - s_{1})}{(s_{2} -  r) (s_{2} - r_{1}) (s_{2} - r_{2}) (s_{2} - s_{1})} \\
    & \qquad \qquad \qquad \qquad + 3 * \frac{ (x  - r_{1}) (x  - r_{2}) (x  - s_{1}) (x  - s_{2})}{(r  - r_{1}) (r  - r_{2}) (r  - s_{1}) (r  - s_{2})}
\end{align*}

\begin{align*}
    v^{'}_{1}(x) &= 1 * \frac{(x  -  r) (x  - r_{2}) (x  - s_{1}) (x  - s_{2})}{(r_{1} -  r) (r_{1} - r_{2}) (r_{1} - s_{1}) (r_{1} - s_{2})}
\end{align*}

\begin{align*}
    v^{'}_{2}(x) &= 1 * \frac{(x  -  r) (x  - r_{1}) (x  - s_{1}) (x  - s_{2})}{(r_{2} -  r) (r_{2} - r_{1}) (r_{2} - s_{1}) (r_{2} - s_{2})}
\end{align*}

\noindent With Lagrange basis polynomials:

\begin{align*}
    l_{r}(x) &= \frac{ (x  - r_{1}) (x  - r_{2}) (x  - s_{1}) (x  - s_{2})}{(r  - r_{1}) (r  - r_{2}) (r  - s_{1}) (r  - s_{2})}, \\
    l_{r_{1}}(x) &= \frac{(x  -  r) (x  - r_{2}) (x  - s_{1}) (x  - s_{2})}{(r_{1} -  r) (r_{1} - r_{2}) (r_{1} - s_{1}) (r_{1} - s_{2})},
    \qquad l_{r_{2}}(x) = \frac{(x  -  r) (x  - r_{1}) (x  - s_{1}) (x  - s_{2})}{(r_{2} -  r) (r_{2} - r_{1}) (r_{2} - s_{1}) (r_{2} - s_{2})} \\
    l_{s_{1}}(x) &= \frac{(x  -  r) (x  - r_{1}) (x - r_{2}) (x  - s_{2})}{(s_{1} -  r) (s_{1} - r_{1}) (s_{1} - r_{2}) (s_{1} - s_{2})},
    \qquad l_{s_{2}}(x) = \frac{(x  -  r) (x  - r_{1}) (x  - r_{2}) (x  - s_{1})}{(s_{2} -  r) (s_{2} - r_{1}) (s_{2} - r_{2}) (s_{2} - s_{1})} \\
\end{align*}

\noindent So, the first set can be rewritten as interpolation polynomials in Lagrange form:

\begin{align*}
    v^{'}_{0}(x) &= 1 * l_{s_{1}}(x) + 1 * l_{s_{2}}(x) + 3 * l_{r}(x) \\
    v^{'}_{1}(x) &= 1 * l_{r_{1}}(x) \\
    v^{'}_{2}(x) &= 1 * l_{r_{2}}(x)
\end{align*}

\noindent Next, rewrite each of the basis polynomials in a form that has a factor
of $ x - r $:

\begin{align*}
    l_{r}(x) &= \frac{(x  - r_{1}) (x  - r_{2}) (x  - s_{1}) (x  - s_{2})}{(r  - r_{1}) (r  - r_{2}) (r  - s_{1}) (r  - s_{2})} \\
    &= \frac{(x^{2} + (-r_{1} - r_{2}) x + r_{1} r_{2}) (x^{2} + (-s_{1} - s_{2}) x + s_{1} s_{2})}{(r  - r_{1}) (r  - r_{2}) (r  - s_{1}) (r  - s_{2})}
\end{align*}

% \begin{align*}
%     &= \frac{((x + (-r1 - r2)x + r1 r2)(x - r) + (r^{2} + (-r1 - r2) r + r1 r2))((x + (-r1 - r2)x + r1 r2)(x - r) + (r^{2} + (-r1 - r2) r + r1 r2))}{(r  - r) (r  - r2) (r  - s1) (r  - s2)} \\
%     &=
% \end{align*}

\noindent Taking the first term in the product of the numerator shows that:

\[
    (x + (-r_{1} - r_{2}) x + r_{1} r_{2}) = (x - r1) (x - r2) (x - r) + (r - r1) (r - r2)  
\]

\noindent Substituting back into the basis polynomial, expanding terms, and consolidating coefficients of $ x - r $
gives:

\begin{align*}
    l_{r}(x) &= \frac{(x  - r_{1}) (x  - r_{2}) (x  - s_{1}) (x  - s_{2})}{(r  - r_{1}) (r  - r_{2}) (r  - s_{1}) (r  - s_{2})} \\
    &= \frac{(x^{2} + (-r_{1} - r_{2}) x + r_{1} r_{2}) (x^{2} + (-s_{1} - s_{2}) x + s_{1} s_{2})}{(r  - r_{1}) (r  - r_{2}) (r  - s_{1}) (r  - s_{2})} \\
    &= \frac{((x - r1) (x - r_{2}) (x - r) + (r - r_{1}) (r - r_{2}))((x - s_{1}) (x - s_{2}) (x - r) + (r - s_{1}) (r - s_{2}))}{(r  - r_{1}) (r  - r_{2}) (r  - s_{1}) (r  - s_{2})} \\
    &=  A (x - r) + B (x - r) + C (x - r) + 1
\end{align*}

\noindent Where:

\begin{align*}
    A &= (x - r_{1}) (x - r_{2}) (x - s_{1}) (x - s_{2}) (x - r) \\
    B &= (r - r_{1}) (r - r_{2}) (x - s_{1}) (x - s_{2}) \\
    C &= (x - r_{1}) (x - r_{2}) (r - s_{1}) (r - s_{2})
\end{align*}

\noindent To make the algebra easier factor out the $ x - r $ term in the basis polynomials:

\begin{align*}
    v^{'}_{0}(x) &= 1 * l^{'}_{s_{1}}(x) (x - r) + 1 * l^{'}_{s_{2}}(x) (x - r) + 3 (A(x) + B(x) + C(x)) (x - r) + 3 \\
    v^{'}_{1}(x) &= 1 * l^{'}_{r_{1}}(x) (x - r) \\
    v^{'}_{2}(x) &= 1 * l^{'}_{r_{2}}(x) (x - r)
\end{align*}

\noindent The other two sets of polynomials become:

\begin{align*}
    w^{'}_{0}(x) &= 1 * \frac{(x  -  r) (x  - r_{2}) (x  - s_{1}) (x  - s_{2})}{(r_{1} -  r) (r_{1} - r_{2}) (r_{1} - s_{1}) (r_{1} - s_{2})} + 1 * \frac{(x  -  r) (x  - r_{1}) (x  - s_{1}) (x  - s_{2})}{(r_{2} -  r) (r_{2} - r_{1}) (r_{2} - s_{1}) (r_{2} - s_{2})} \\
    &= 1 * l_{r_{1}}(x) + 1 * l_{r_{2}}(x) \\
    &= 1 * l^{'}_{r_{1}}(x) (x - r) + 1 * l^{'}_{r_{2}}(x) (x - r)
\end{align*}

\begin{align*}
    w^{'}_{1}(x) &= 1 * \frac{(x  -  r) (x  - r_{1}) (x  - r_{2}) (x  - s_{2})}{(s_{1} -  r) (s_{1} - r_{1}) (s_{1} - r_{2}) (s_{1} - s_{2})} + 1 * \frac{(x  - r1) (x  - r_{2}) (x  - s_{1}) (x  - s_{2})}{(r  - r_{1}) (r  - r_{2}) (r  - s_{1}) (r  - s_{2})} \\
    &= 1 * l_{s_{1}}(x) + 1 * l_{r}(x) \\
    &= 1 * l^{'}_{s_{1}}(x) (x - r) + (A(x) + B(x) + C(x)) (x - r) + 1
\end{align*}

\begin{align*}
    w^{'}_{2}(x) &= 1 * \frac{(x  -  r) (x  - r_{1}) (x  - r_{2}) (x  - s_{1})}{(s_{2} -  r) (s_{2} - r1) (s_{2} - r_{2}) (s_{2} - s_{1})} \\
    &= 1 * l_{s_{2}}(x) \\
    &= 1 * l^{'}_{s_{2}}(x) (x - r)
\end{align*}

\noindent and

\begin{align*}
    y^{'}_{0}(x) &= 0
\end{align*}

\begin{align*}
    y^{'}_{1}(x) &= 1 * \frac{(x  -  r) (x  - r_{2}) (x  - s_{1}) (x  - s_{2})}{(r_{1} -  r) (r_{1} - r_{2}) (r_{1} - s_{1}) (r_{1} - s_{2})} + 1 * \frac{(x  -  r) (x  - r_{1}) (x  - r_{2}) (x  - s_{2})}{(s_{1} -  r) (s_{1} - r_{1}) (s_{1} - r_{2}) (s_{1} - s_{2})} \\
    &= 1 * l_{r_{1}}(x) + 1 * l_{s_{1}}(x) \\
    &= 1 * l^{'}_{r_{1}}(x) (x - r) + 1 * l^{'}_{s_{1}}(x) (x - r)
\end{align*}

\begin{align*}
    y^{'}_{2}(x) &= 1 * \frac{(x  - r_{1}) * (x  - r_{2}) * (x  - s_{1}) * (x  - s_{2})}{(r  - r_{1}) * (r  - r_{2}) * (r  - s_{1}) * (r  - s_{2})} \\
    &+ 1 * \frac{(x  -  r) (x  - r_{1}) (x  - s_{1}) (x  - s_{2})}{(r_{2} -  r) (r_{2} - r_{1}) (r_{2} - s_{1}) (r_{2} - s_{2})} \\
    &+ 1 * \frac{(x  -  r) (x  - r_{1}) (x  - r_{2}) (x  - s_{1})}{(s_{2} -  r) (s_{2} - r_{1}) (s_{2} - r_{2}) (s_{2} - s_{1})} \\
    &= 1 * l_{r}(x) + 1 * l_{r_{2}}(x) + 1 * l_{s_{2}}(x) \\
    &= 1 * l^{'}_{r_{2}}(x) (x - r) + 1 * l^{'}_{s_{2}}(x) (x - r) + (A(x) + B(x) + C(x)) (x - r) + 1
\end{align*}

\noindent Now, before calculating the full expression simplify each of the terms with a factor $ x - r $:

\begin{align*}
    v_{0} + \Sigma_{k = 1}^{2} a_{k} v_{k} &= 1 * l_{s_{1}}(x) + 1 * l_{s_{2}}(x) + 3 * l_{r}(x) + a_{1} * l_{r_{1}}(x) + a_{2} * l_{r_{2}}(x) \\
    &= 3 (A(x) + B(x) + C(x)) (x - r) + 3 \\
    &+ a_{1} * l^{'}_{r_{1}}(x) (x - r) + a_{2} * l^{'}_{r_{2}}(x) (x - r) + 1 * l^{'}_{s_{1}}(x) (x - r) + 1 * l^{'}_{s_{2}}(x) (x - r) \\
    &= (3 A(x) + 3 B(x) + 3 C(x) + a_{1} * l^{'}_{r_{1}}(x) + a_{2} * l^{'}_{r_{2}}(x) + 1 * l^{'}_{s_{1}}(x) + 1 * l^{'}_{s_{2}}(x)) (x - r) + 3 \\
    &= f_{v}(x) (x - r) + 3
\end{align*}

\begin{align*}
    w_{0} + \Sigma_{k = 1}^{2} a_{k} w_{k} &= 1 * l_{r_{1}}(x) + 1 * l_{r_{2}}(x) + a_{1} * l_{s_{1}}(x) + a_{1} * l_{r}(x) + a_{2} * l_{s_{2}}(x) \\
    &= a_{1} (A(x) + B(x) + C(x)) (x - r) + a_{1} \\
    &+ 1 * l^{'}_{r_{1}}(x) (x - r) + 1 * l^{'}_{r_{2}}(x) (x - r) + a_{1} * l^{'}_{s_{1}}(x) (x - r) + a_{2} * l^{'}_{s_{2}}(x) (x - r) \\
    &= (a_{1} A(x) + a_{1} B(x) + a_{1} C(x) + 1 * l^{'}_{r_{1}}(x) + 1 * l^{'}_{r_{2}}(x) + a_{1} * l^{'}_{s_{1}}(x) + a_{2} * l^{'}_{s_{2}}(x))(x - r) + a_{1} \\
    &= f_{w}(x) (x - r) + a_{1}
\end{align*}

\begin{align*}
    y_{0} + \Sigma_{k = 1}^{2} a_{k} y_{k} &= a_{1} * l_{r_{1}}(x) + a_{1} * l_{s_{1}}(x) + a_{2} * l_{r}(x) + a_{2} * l_{r_{2}}(x) + a_{2} * l_{s_{2}}(x) \\
    &= a_{2} (A(x) + B(x) + C(x)) (x - r) + a_{2} \\
    &+ a_{1} * l^{'}_{r_{1}}(x) (x - r) + a_{1} * l^{'}_{s_{1}}(x) (x - r) + a_{2} * l^{'}_{r_{2}}(x) (x - r) + a_{2} * l^{'}_{s_{2}}(x) (x - r) \\
    &= f_{y}(x)(x - r) + a_{2}
\end{align*}

\noindent The product is then:

\begin{align*}
    (v_{0} + \Sigma_{k = 1}^{2} a_{k} v_{k}) (w_{0} + \Sigma_{k = 1}^{2} a_{k} w_{k}) - (y_{0} + \Sigma_{k = 1}^{2} a_{k} y_{k}) &= (f_{v}(x) (x - r) + 3)(f_{w}(x) (x - r) + a_{1}) - (f_{y}(x)(x - r) + a_{2}) \\
    &= f_{v} (x) f_{w}(x) (x - r)^{2} + a_{1} f_{v}(x) (x - r) + 3 f_{w}(x)(x - r) + 3 a_{1} \\
    &-f_{y}(x)(x - r) - a_{2} \\
    &= f_{v} (x) f_{w}(x) (x - r)^{2} + 2 f_{v}(x) (x - r) + 3 f_{w}(x)(x - r) + 3 * 2 \\
    &-f_{y}(x)(x - r) - 6 \\
    &= (f_{v} (x) f_{w} (x) (x - r) + 2 f_{v}(x) + 3 f_{w}(x) - f_{y(x)})(x - r)
\end{align*}

\noindent An expression of the form $ h(x) (x - r) $ as needed.

\subsection*{Zero Knowledge Set Membership}

\noindent Derivation showing the equality between what the prover generates and
what the verifier checks. From the construction:

\[
    y = g^{x}, \quad A_{i} = g^{\frac{1}{x + i}}, \quad V = A_{\delta}^{\tau} = g^{\frac{\tau}{x + \delta}} 
\]

\noindent Prover claims to send:

\[
    a = e(V, g)^{-s} \cdot e(g, g)^{t}
\]

\noindent and to validate this the verifier send $ c $ for which if they recieve (supposedly)
$ z_{\delta} = s - \delta c $, $ z_{\tau} = t - \tau c $, and
$ z_{\gamma} = m - \gamma c $. So they can check the validity of $ a $ because:

\begin{align*}
    e(V, y)^{c} \cdot e(V, g)^{-z_{\delta}} \cdot e(g, g)^{z_{\tau}} &=
    e \left(g^{\frac{\tau}{x + \delta}}, g^{x} \right)^{c} \cdot e \left(g^{x}, g \right)^{-z_{\delta}} \cdot e(g, g)^{z_{\tau}} \\
    &= e(g, g)^{\frac{c \tau x}{x + \delta}} \cdot e(g, g)^{-z_{\delta} x} \cdot e(g, g)^{z_{\tau}} \\
    &= e(g, g)^{\frac{c \tau x}{x + \delta} - z_{\delta} x + z_{\tau}}
\end{align*}

\noindent the prover had sent:

\begin{align*}
    e(V, g)^{-s} \cdot e(g, g)^{t} &= e \left(g^{\frac{\tau}{x + \delta}}, g \right)^{-s} \cdot e(g, g)^{t} \\
    &= e(g, g)^{\frac{-s \tau}{x + \delta}} \cdot e(g, g)^{t} \\
    &= e(g, g)^{\frac{-s \tau}{x + \delta} + t}
\end{align*}

\noindent So, if the prover did in fact supply correct $ z_{\delta} $,
$ z_{\tau} $, and $ z_{\gamma} $ after the verifier shared $ c $ the two
expressions should be equal because the exponent of the verifier's calculation
would evaluate to:

\begin{align*}
    \frac{c \tau x}{x + \delta} - z_{\delta} x + z_{\tau} &= \frac{c \tau x - \tau z_{\delta} + (x + \delta) z_{\tau}}{x + \delta} \\
    &= \frac{c \tau x - \tau (s - \delta c) + (x + \delta) (t - \tau c)}{x + \delta} \\
    &= \frac{c \tau x - \tau s + \tau \delta c + t (x + \delta) - \tau c (x + \delta)}{x + \delta} \\
    &= \frac{t(x + \delta) - \tau s}{x + \delta} \\
    &= \frac{-s \tau}{x + \delta} + t
\end{align*}

\end{document}
